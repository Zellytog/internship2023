\documentclass{article}
\usepackage[utf8]{inputenc}
\usepackage[english]{babel}
\usepackage{amsmath,amssymb,bm, amsthm}
\usepackage{color}
\usepackage{tikz}
\usepackage{tikz-cd}
\usepackage{version}
\usepackage[T1]{fontenc}
\usepackage{graphicx}
\usepackage{bbm}

\newtheorem{defi}{Definition}
\newtheorem{lem}[defi]{Lemma}
\newtheorem{them}[defi]{Theorem}
\newtheorem{expl}[defi]{Example}
\newtheorem{prop}[defi]{Proposition}
\newtheorem{cor}{Corrolary}[defi]
\newtheorem{rmk}{Remark}

\makeatletter
\newcommand*\bigcdot{\mathpalette\bigcdot@{.5}}
\newcommand*\bigcdot@[2]{\mathbin{\vcenter{\hbox{\scalebox{#2}{$\m@th#1\bullet$}}}}}
\makeatother

\newcommand{\toa}[0]{\to_\mathrm a}
\newcommand{\btwo}[0]{\mathbbm{2}}
\newcommand{\twotors}[0]{2\mathrm{Tors}}
\newcommand{\idd}[0]{\mathrm{id}}
\newcommand{\refl}[0]{\mathrm{refl}}
\newcommand{\oplusa}[0]{\oplus_\mathrm a}

\title{A formalisation of the sign in HoTT}
\author{Titouan Leclercq}
\date{June 2023}

\begin{document}

\maketitle

\section{Introduction}

This is a document written in the context of an internship at Chalmers university. The purpose of this
internship is to construct a new definition of the sign of a permutation in Homotopy Type Theory (HoTT). A
formal proofs in Agda are quite hard too read, this document explains the main ideas behind the construction
and tries to give an intuition of the results. We assume some familiarity with HoTT.

The construction is made in the following steps :
\begin{itemize}
    \item Construction of the group $\twotors$ of $2$-elements sets.
    \item Construction of a sum indexed by a finite set in a group, using the notion of biproduct in
    \textbf{Ab}.
    \item Definition of the orientation of a finite set and construction of the sign.
\end{itemize}

We will thus explain all the steps, and begin by defining the notion of group and more precisely abelian
group, as they are not seen here as sets with a binary product but as spaces.

\textit{NB}: in the formalised work in Agda we keep count of the universe, but for a concision purpose we will
just use $\mathcal U$ for any universe here.

\section{What is a group?}

In HoTT, two definitions can be considered for a group.

\subsection{A set with a convenient operation}

The first one, which is the usual one in classical
maths, is the definition of a set with inversible associative operation and an identity element.

\begin{defi}[Abstract group]
    We define the type of abstract groups as a set $X$ with a group structure, i.e. a binary operation, a
    distinguished point and the usual properties of a group, described by the following types:
    
    \begin{align*}
        &\mathrm{assoc} : \prod_{X : \mathcal U} (X\to X \to X) \to \mathcal U\\
        &\mathrm{assoc}\;X\;(\cdot) :\equiv \prod_{x\;y\;z : X} ((x\cdot y)\cdot z = x\cdot (y\cdot z))\\ \\
        &\mathrm{id}_\mathrm G : \prod_{X : \mathcal U} (X \to X \to X) \to X \to \mathcal U\\
        &\mathrm{id}_\mathrm G\;X\;(\cdot)\;e :\equiv \prod_{x : X} (x\cdot e = x) \times (e \cdot x = x)\\ \\
        &\mathrm{inv} : \prod_{X : \mathcal U} (X \to X \to X) \to X \to \mathcal U\\
        &\mathrm{inv} \;X\;(\cdot)\;e :\equiv \prod_{x : X} \sum_{x' : X} ((x\cdot x' = e) \times (x'\cdot x 
        = e))\\
        \\
        &\mathrm{GrpStr} : \mathcal U \to \mathcal U\\
        &\mathrm{GrpStr}\;X :\equiv \left(\sum_{e : X} \sum_{\cdot : X \to X \to X} 
        (\mathrm{assoc}\;X\;(\cdot))\times (\mathrm{id}_\mathrm G \;X\;(\cdot)\;e)\times
        (\mathrm{inv}\;X\;(\cdot)\;e)\right)\\ \\
        &\mathrm{isGrp} : \mathcal U \to \mathcal U\\
        &\mathrm{isGrp}\;X :\equiv (\mathrm{isGrp}\;X)\times(\mathrm{GrpStr}\;X)
    \end{align*}
\end{defi}

\begin{rmk}
    We could also define a function $i$ instead of making a $\Sigma$-type for $\mathrm{inv}$.
\end{rmk}

We will call \textit{abstract group} such a group, and the predicate isGrp is now renamed as isSetGrp, as this
notion of group is a group structure on a set.

\subsection{A connected pointed groupoid}

However, another way of representing a group is to consider the Eilenberg-MacLane space of it. For example
instead of considering $(\mathbb Z,+,0)$ we can just consider $(\mathbb S^1,\mathrm{base})$. This point of
view is developped in the Symmerty book, and we can prove that \textbf{Grp} is equivalent (thus equal) to
\textbf{GrpSpc}, the category of $K(G,1)$ for each $G : \textbf{Grp}$. For this work, we will thus consider
\textbf{Grp} as the type of spaces corresponding to some $K(G,1)$. The question of characterizing those spaces
is answered in the symmerty book by the notion of connected pointed groupoid.

\begin{defi}[Group]
    A group is a space $X$ with a distinguished point $x_0$ which is connected and is a $1$-type. This can be
    summarized by the following types:
    
    \begin{align*}
        &\mathrm{connected} : \mathcal U \to \mathcal U\\
        &\mathrm{connected}\;X :\equiv \prod_{x\;y : X} \left\| x = y\right\|_1\\
        &\mathrm{isGrpoid} : \mathcal U \to \mathcal U\\
        &\mathrm{isGrpoid}\;X :\equiv \mathrm{is-}n\mathrm{-type}\;1\;X\\
        &\mathrm{isGrp} : \mathcal U \to X \to \mathcal U\\
        &\mathrm{isGrp}\;X\;x_0 :\equiv (\mathrm{connected\;X})\times(\mathrm{isGrpoid}\;X)
    \end{align*}

    We also define $$\textbf{Grp} :\equiv \sum_{X : \mathcal U}\sum_{x_0 : X} \mathrm{isGrp}\;X\;x_0$$
\end{defi}
    
For a space $(X,x_0)$ its corresponding abstract group is $\pi_1(X,x_0) \equiv (x_0=x_0,\mathrm{refl}_{x_0})$.

\subsection{Abelian group}

Now we focus on abelian groups, also as spaces. For abstract groups, being abelian just means that for all
$x,y : X$ we have $x\cdot y = y\cdot x$, but in the case of spaces the proposition we add is a structure
defined uniquely.

\begin{defi}[Homogeneous pointed type]
    A pointed type $(X,x_0)$ is homogeneous if, for all $x : X$ we have a path $(X,x_0)= (X,x)$ which can
    equivalently be formulated as giving for each $x : X$ an equivalence $e : X \simeq X$ such that 
    $e(x_0) = x$. We say that $(X,x_0)$ is pointed-homogeneous if, moreover, there is an homogeneity $h$ such
    that $h\;x_0 = \mathrm{refl}_{(X,x_0)}$.
    \begin{align*}
        &\mathrm{Homogeneous} : \prod_{X : \mathcal U} X \to\mathcal U\\
        &\mathrm{Homogeneous}\;X\;x_0 :\equiv \prod_{x : X} ((X,x_0) = (X,x))\\
        &\mathrm{Homogeneous}\bullet : \prod_{X : \mathcal U} \prod_{x_0 : X} \mathrm{Homogeneous}\;X\;x_0 \to
        \mathcal U\\
        &\mathrm{Homogeneous}\bullet\;X\;x_0\;h :\equiv h = \mathrm{refl}_{(X,x_0)}\\
        &\mathrm{isHomo} : \prod_{X : \mathcal U} X \to\mathcal U\\
        &\mathrm{isHomo}\;X\;x_0 :\equiv \sum_{h : \mathrm{Homogeneous}\;X\;x_0} 
        (\mathrm{Homogeneous}\bullet\;X\;x_0\;h)
    \end{align*}
\end{defi}

The type Homogeneous is, by itself, not a proposition. However by imposing that it is refl on the basepoint,
this type becomes a proposition. The proof use an essential lemma:

\begin{lem}
    If $X$ is a $n$-connected type and $B : X \to \mathcal U$ is a family of $n$-types, then given $x_0 : X$
    and $b_0 : B(x_0)$ the following type is a $n-1$-type:
    $$\sum_{f : \prod_{x : X}B(x)} (f(x_0)=b_0)$$
\end{lem}

\begin{proof}
    We prove this lemma by induction on $n$. 
    \begin{itemize}
        \item If $B(x)$ is a proposition then we can prove that it is 
        contractible using the fact that $X$ is connected, by transporting $b_0$ through $B$, along the
        equalities $\|x_0=x\|_1$ (as $B$ is a family of propositions we can use the induction principle of
        propositional truncation), to give us the center of contraction. As each $B(x)$ is a proposition, this
        is enough to show thee contractibility.
        \item If we take $(f,f_\bullet,(g,g_\bullet)$ in our type supposed to be a $n-1$-type, 
        we want to show that $(f,f_\bullet) = (g,g_\bullet)$ is a $n$-type. An equality in this type is
        equivalently a pair of a homotopy $h : \displaystyle \prod_{x : X} (f\;x = g\;x)$ and $f_\bullet = g_\bullet$
        above $h$.
        We can also say that the following diagramm commutes :
        \begin{center}
        \begin{tikzcd}
            f\;x_0\ar[dr,"f_\bullet"]\ar[rr,"h_{x_0}"] & & g\;x_0 \ar[dl,"g_\bullet"]\\
            &b_0
        \end{tikzcd}
        \end{center}

        So $(f,f_\bullet) = (g,g_\bullet)$ is equivalent to the type 
        $$\sum_{h : \prod_{x : X} f\;x=g\;x}(h\;x_0 = f_\bullet\bigcdot g_\bullet)$$
        on which we can apply the induction hypothesis to conclude.
    \end{itemize}
\end{proof}

\begin{cor}
    The type $\mathrm{isHomo}\;X\;x_0$ is a proposition.
\end{cor}

\begin{proof}
    First, for $x : X$, the type $(X,x_0)=(X,x)$ is equivalent to 
    $$\sum_{(f,f_\simeq) : X \simeq X}(f(x_0)=x)$$
    and $X$ is a connected $1$-type, so $(X,x_0)=(X,x)$ is a set. Now isHomo can be seen as the type
    $$\sum_{f : \prod_{x : X} B(x)} (f(x_0)=b_0)$$
    for $B(x) :\equiv (X,x_0)=(X,x)$ and $b_0 :\equiv \mathrm{refl}_{(X,x_0)}$, so this type is a $-1$-type,
    i.e. a proposition.
\end{proof}

\begin{defi}[The category \textbf{Ab}]
    We define the objects as the following type:

    $$\textbf{Ab} :\equiv \sum_{X : \mathcal U}\sum_{x_0 : X} (\mathrm{isGrp}\;X\;x_0)\times 
        (\mathrm{isHomo}\;X\;x_0)$$

    For $X : \textbf{Ab}$ we will write $(\lceil X \rfloor,x_0,g_X,h_X,h_{X\bullet})$ for its different
    components. We also define the type of arrows between $X Y : \textbf{Ab}$ as follows:

    $$X\to_\mathrm a Y :\equiv \sum_{f : \lceil X \rfloor \to \lceil Y\rfloor} (f\;x_0 = y_0)$$
    
\end{defi}

A first result is that $\textbf{Ab}$ is indeed a category.

\begin{prop}
    The type $X\to_\mathrm a Y$ is a set.
\end{prop}

\begin{proof}
    It is also an application of the previous lemma, taking $B(x) :\equiv \lceil Y \rfloor$
    and $b_0 :\equiv y_0$.
\end{proof}

This category is univalent, in the sense that having an equivalence of pointed type between two groups is
the same as having an equality in \textbf{Grp} (this is because the other parts of this type are
propositions).

\subsection{Addition in an abelian group}

We conclude this section on the notion of space-addition. The homogeneity gives us a natural notion of
addition.

\begin{defi}
    Given $X : \textbf{Ab}$ we define the following operation :
    \begin{align*}
        &\_+_X\_ : \lceil X \rfloor \to \lceil X \rfloor \to \lceil X \rfloor \\
        &x+_Xy :\equiv \mathrm{transport}^{(\lambda (X,x_0).X)}(h_X\;x,y)
    \end{align*}
\end{defi}

We write it as an addition because it has the properties of a group. The difference, however,
is that $X$ is not a set. The proof of associativity, commutativity and so on relies on the fact that we can
define a good type of addition as follows:
    $$\prod_{f : (x : \lceil X \rfloor) \to X \to_\mathrm a X'} (f\; x_0 = \mathrm{id}_X)$$
where $X'$ is $X$ but with $x$ as its basepoint. This type can be easily shown to be a proposition, and both
$(x,y)\mapsto x +_X y$ and $(x,y)\mapsto y+_Xx$ are in this type. In a similar way we can define a ternary
sum and thus show that $+_X$ is associative. We give a final result regarding the addition:

\begin{prop}
    Let $X\;Y : \textbf{Ab}$ and $f : X \toa Y$. Then for all $x\;y : X$ we have
    $$f(x+_Xy) = f(x)+_Yf(y)$$
\end{prop}

\begin{proof}
    Given $(f,f_\bullet) : X \toa Y$, we define the following type:
    $$F:\equiv \sum_{g : \prod_{x : X} X \toa Y'} (g(x_0) = (f,\mathrm{refl}_{f\;x_0})$$ 
    where $Y'$ is $Y$ with basepoint set at $f\;x$.

    $F$ is a proposition. Indeed, with the previous lemma, we juste have to show that 
    $$\prod_{x : X}X\toa Y'$$ is a set, but $X \toa Y'$ is a set as we proved before, and a family os sets
    is always a set, this type is a set.

    Now we can see that both $(x,y)\mapsto f(x+_Xy)$ and $(x,y)\mapsto f(x)+_Yf(y)$ can be proved to belong to
    $F$, so they are equal, which is equivalent to the homotopy 
    $$\prod_{x\;y : X}(f(x+_Xy)=f(x)+_Yf(y))$$
\end{proof}

\section{The $\mathbb Z/2\mathbb Z$ torsors}

A way of constructing the Eilenberg-MacLane space of a group $G$ is to study the type of $G$-torsors. We will
not consider the $G$-torsors in the general case but in the case of $G = \mathbb Z/2\mathbb Z$. In this case,
the type can be described easily as the type of sets with $2$ elements. To be more concise, we will speak of
$2$-torsors instead of $\mathbb Z/2\mathbb Z$-torsors.

\begin{defi}
    We define the type of $2$-torsors as such :
    $$2\mathrm{Tors} := \sum_{X : \mathcal U} \left\| X \simeq \btwo \right\|_1$$
\end{defi}

\begin{rmk}
    It is essential that we have a propositional truncation here, because otherwise it would just be a
    singleton of $\btwo$.
\end{rmk}

\subsection{$\twotors$ is a group}

Some properties easily follow from the definition.

\begin{prop}
    The type $\twotors$ is pointed, connected and is a groupoid.
\end{prop}

\begin{proof}
    The point is obviously $\btwo$ : $$(\btwo , |\idd_\btwo |_1) : \twotors$$

    For $X\;Y : \twotors$, we can reason by induction on $\|X\simeq\btwo\|_1$ and $\|Y\simeq\btwo\|_1$ because
    $\|X\simeq Y\|_1$ is also a proposition, and we use the following function:
    $$\lambda (e : X\simeq\btwo).\lambda (e' : Y\simeq\btwo). |e\bigcdot e'^{-1}|_1 : X\simeq\btwo \to
    Y\simeq \btwo \to \|X \simeq Y\|_1$$

    For $(X,e) : \twotors$ we know that $\mathrm{isSet}\;X$ because $X \simeq \btwo$ in regard to
    propositions. As a family of sets is also a set, for $(X,e)\;(Y,e') : \twotors$, the type $X = Y$ is
    also a set, and as $\|X\simeq \btwo\|_1$ is a proposition, $((X,e)=(Y,e'))\simeq X=Y$ so the equalities in
    $\twotors$ are a set. So $\twotors$ is a $1$-type, i.e. a groupoid.
\end{proof}

Thus $\twotors : \textbf{Grp}$.

Let's give an important result to describe the structure of $\twotors$.

\begin{prop}
    $(\btwo\simeq\btwo) \simeq \btwo$
\end{prop}

\begin{cor}
    $\Omega (\twotors,(\btwo , |\idd_\btwo|_1)) \simeq (\btwo,\oplus)$
\end{cor}

\begin{proof}
    As $(\btwo = \btwo)\simeq (\btwo\simeq\btwo)$ and $(\btwo\simeq\btwo)\simeq\btwo$, we conclude the result.
\end{proof}

In fact, for all $X : \twotors$, we can define the action of $\btwo$ on it. The intuition is that $0$ acts as
the identity and $1$ as the permutation of the two elements of $X$.

\begin{prop}
    For $(X,e) : \twotors$ we define the action $\bigcdot$ of $\btwo$ on $X$. It is a group action of 
    $(\btwo,\oplus)$ on $X$ and it is moreover transitive.
\end{prop}

\subsection{Sum of torsors}

To show that $\twotors$ is homogeneous, we construct an operation on the torsors, which we will call a sum.
To construct it, we define the type of structures on two $2$-torsors which commute with the action.

\begin{defi}
    Let $X\;Y : \twotors$. We define the following types:
    \begin{align*}
        &\mathrm{bilinl} : \prod_{Z : \twotors} (X\to Y\to Z) \to \mathcal U\\
        &\mathrm{bilinl}\;Z\;(+_Z) :\equiv \prod_{x : X}\prod_{y : Y}\prod_{b : \btwo}
        ((x\bigcdot b) +_Z y = (x+_Z y)\bigcdot b)\\
        &\mathrm{bilinr} : \prod_{Z : \twotors} (X\to Y\to Z) \to \mathcal U\\
        &\mathrm{bilinr}\;Z\;(+_Z) :\equiv \prod_{x : X}\prod_{y : Y}\prod_{b : \btwo}
        (x+_Z (y\bigcdot b) = (x+_Z y)\bigcdot b)\\
        &\_+_{\twotors}\_ : \mathcal U\\
        &X+_{\twotors} Y :\equiv \sum_{Z : \twotors}\;\;\sum_{(+_Z) : X \to Y \to Z} 
        (\mathrm{bilinl}\;Z\;(+_Z))\times (\mathrm{bilinr}\;Z\;(+_Z))
    \end{align*}
\end{defi}

\begin{them}
    For $X\;Y : \twotors$, the type $X+_{\twotors} Y$ is contractible.
\end{them}

\begin{proof}
    Showing that a type is contractible is a proposition. This means that, to prove this for each $X\;Y : 
    \twotors$, it suffices to show it for $\btwo+_{\twotors}\btwo$. It is straightforward to check that the
    pair $(\btwo,\oplus)$ is such an element, and for any $(Z,+_Z)$ in $\btwo+_{\twotors}\btwo$ we can make
    the equivalence $\_+_Z 0 : \btwo \simeq Z$ because $\bigcdot$ is transitive and we can rewrite
    $b+_Z 0$ as $(0+_Z0)\bigcdot b$, it is then left to check that this equivalence commutes with the action,
    which it does.
\end{proof}

We will thus write $X+_{\twotors} Y$ for the torsor in the center of contraction of the associated type.

This sum makes $\twotors$ into a monoidal symmetric category and $X$ is its own inverse.

\begin{prop}
    The following properties hold for $X\;Y : \twotors$:
    \begin{itemize}
        \item $X+_{\twotors}\btwo = X$
        \item $X+_{\twotors} Y = Y+_{\twotors} X$
        \item $X+_{\twotors} X = \btwo$
        \item $X+_{\twotors} (Y+_{\twotors} Z) = (X+_{\twotors} Y)+_{\twotors} Z$
    \end{itemize}
\end{prop}

\begin{proof}
    We can see that $\bigcdot$ is a bilniear function $X \to \btwo\to X$ so by contractibility of the type of
    sum, the equality follows.

    We take $(Y+_{\twotors} X , \lambda x.\lambda y. y+x)$ to be in the type of the sum of $X$ and $Y$, hence
    the equality.

    We associate $(x,y)$ to $0$ and $(x,y), x\neq y$ to $1$ and check that this function is indeed bilinear.

    We construct the type of ternary sums, in a similar way as for the binary sum, and show it is a 
    proposition. They are both ternary sums so they are equal.
\end{proof}

This leads to the homogeneity.

\begin{them}
    The type $\twotors$ is homogeneous.
\end{them}

\begin{proof}
    Given $X : \twotors$, we define 
    \begin{align*}
        &h : \twotors \to \twotors\\
        &h_X\;Z :\equiv Z+_{\twotors}X
    \end{align*}

    This is an equivalence because it is involutive : 
    \begin{align*}
        (Z+_{\twotors} X)+_{\twotors} X &= Z+_{\twotors}(X+{\twotors} X)\\
        &= Z+_{\twotors}\btwo\\
        &= Z
    \end{align*}
    and $\btwo + X = X$ so $\mathrm{ua}(h_X) : (\twotors, \btwo) = (\twotors , X)$, and $h_\btwo$ is the
    function $X\mapsto X+\btwo$ which is $X\mapsto X$, so $\mathrm{ua}(h_\btwo) = \refl_{(\twotors, \btwo)}$.
\end{proof}

This means that $\twotors : \textbf{Ab}$.

\section{Defining a sum indexed by a finite set}

The purpose of this section is, for an abelian group $A$, to define an operator $\Sigma^I_X$, for a finite set
$J$, taking a family $(x_j)_{j\in J}\in X^J$ to $X$. In the finite case, we could easily define it by
induction by saying that $\Sigma_X^\varnothing x_j :\equiv 0_X$ and that 
$\Sigma_X^{\{1,\ldots,n+1\}} x_j :\equiv \left(\Sigma_X^{\{1,\ldots,n\}} x_j\right) + x_{n+1}$, but this is
dependant on the order in which the elements come. If we want to extend this definition to any finite set,
which is a type with a mere identification with some $\mathrm{Fin}\;n$, we need to construct a sum which is a
proposition. To do so, we will use the categorical language, to construct finite biproducts and then use the
codiagonal.

\subsection{The biproduct}

\subsubsection{The binary biproduct}

The category \textbf{Ab}, as one expects, is abelian. This implies that we can construct a type $A \oplusa B$
for two groups $A$ and $B$ equiped with:
\begin{itemize}
    \item two projections $\pi_1 : A\oplusa B \toa A$ and $\pi_2 : A \oplusa B \toa B$ universal in the sense
    that the associated function $$(X\toa A\oplusa B) \to (X\toa A)\times (X\toa B)$$ is an equivalence for
    all group $X$.
    \item two coprojections $\kappa_1 : A \toa A\oplus B$ and $\kappa_2 : B\toa A \oplusa B$ universal in the
    sens that the associated function $$(A\oplusa B \toa X) \to (A\toa X) \times (B\toa X)$$ is an equivalence
    for all group $X$.
    \item a proof that $[\langle \idd_\mathrm a A, 0_\mathrm a\;A\;B\rangle , \langle 0_\mathrm a\;B\;A,\idd
    _\mathrm a B\rangle] = \idd_{A\oplusa B}$.
\end{itemize}

\begin{defi}
    Let $A\;B : \textbf{Ab}$. We define the following terms for $C : \textbf{Ab}$:
    \begin{align*}
        &\mathrm{Prod}_0\;C :\equiv (C\toa A)\times (C\toa B)\\
        &\mathrm{distr}_\times\;C : \prod_{\pi : \mathrm{Prod}_0\;C} \prod_{X : \textbf{Ab}}
        (X\toa C) \to (X\toa A)\times (X\toa B)\\
        &\mathrm{distr}_\times \;C \; (\pi_1,\pi_2)\;X\;f :\equiv (\pi_1\circ f, \pi_2\circ f)\\
        &\mathrm{univ}_\times\;C\;\pi :\equiv \prod_{X : \textbf{Ab}} \mathrm{isEquiv}\;
        (\mathrm{distr}_\times C\;\pi\;X)\\
        &\mathrm{Prod}\;C : \equiv \sum_{\pi : \mathrm{Prod}_0\;C}(\mathrm{univ}_\times \pi)
    \end{align*}
    \begin{align*}
        &\mathrm{Coprod}_0\;C :\equiv (A\toa C)\times (B\toa C)\\
        &\mathrm{distr}_+\;C : \prod_{\kappa : \mathrm{Coprod}_0\;C} \prod_{X : \textbf{Ab}}
        (C\toa X) \to (A\toa X)\times (B\toa X)\\
        &\mathrm{distr}_+ \;C \; (\kappa_1,\kappa_2)\;X\;f :\equiv (f\circ\kappa_1, f\circ\kappa_2)\\
        &\mathrm{univ}_+\;C\;\kappa :\equiv \prod_{X : \textbf{Ab}} \mathrm{isEquiv}\;
        (\mathrm{distr}_+ C\;\kappa\;X)\\
        &\mathrm{Coprod}\;C : \equiv \sum_{\kappa : \mathrm{Coprod}_0\;C}(\mathrm{univ}_\times C\;\kappa)\\
        &\mathrm{Biprod} :\equiv \sum_{C : \textbf{Ab}}\sum_{p : \mathrm{Prod}\;C \times \mathrm{Coprod}\;C}
        \mathrm{compatibility}
    \end{align*}
    Where compatibility is the property described above showing that $\pi$ and $\kappa$ are compatible.
\end{defi}

\begin{rmk}
    As isEquiv is a proposition and equalities in $\toa$ are propositions (because $\toa$ is a set), the only
    real structures are the carrier $C$, the projections and the coprojections.
\end{rmk}

\begin{them}
    The type $\mathrm{Biproduct}\;A\;B$ is a proposition.
\end{them}

\begin{proof}
    Suppose that there are two biproducts $(X,\pi,\kappa)$ and $(Y,\pi',\kappa')$ of $A$ and $B$. 
    We can define by unicity of the universal property of product an equality $p_\times : X = Y$ which
    commutes with $\pi$ and $\pi'$. Similarly we can construct $p_+ : X = Y$ which commutes with $\kappa$ and
    $\kappa'$. Finaly by a quick computation using the compatibility condition we can find that $p_\times =
    p_+$ so there is an equality which makes everything commute, so the two structures are equal.
\end{proof}

\begin{them}
    The type $\mathrm{Biproduct}\;A\;B$ is contractible.
\end{them}

\begin{proof}
    We prove that there is an inhabitant of this type. The groupe is simply $A\times B$ with the point $(a_0,
    b_0)$, and all the propositions follow directly. We define:
    \begin{itemize}
        \item $\kappa_1 : a \mapsto (a,b_0)$.
        \item $\kappa_2 : b \mapsto (a_0,b)$.
        \item $\pi_1 : (a,b) \mapsto a$.
        \item $\pi_2 : (a,b) \mapsto b$.
    \end{itemize}

    The fact that $\pi_1,\pi_2$ are a product structure is straightforward. For the coproduct structure, we
    take $X : \textbf{Ab}$ and two functions $f : A\toa X$, $g : B \toa X$ and construct the center of the
    fiber $[f,g] : A \oplusa B \toa X$ as $(a,b) \mapsto f(a)+_{A\oplusa B} g(b)$. As the homogeneity acts
    componentwise we can always destruct $(a,b)$ as $(a,b_0)+_{A\oplusa B} (a_0,b)$ and thus for
    $h : A \oplusa B \toa X$ we have $[h\circ \kappa_1,h\circ \kappa_2] : (a,b) \mapsto h(a,b)$ because
    every function commutes with $+$.

    Finally, checking that $[\langle \idd_\mathrm a A,0_\mathrm a\;A\;B\rangle , \langle 0_\mathrm a\;A\;B,
    \idd_\mathrm a\;B\rangle] = \idd_{A\oplusa B}$ is straightforward.
\end{proof}

\subsubsection{Generalizing to a finite family}

We can change slightly the definitions to make a type of biproducts for any finite family of abelian groups.
The proof that the type of such biproducts is a proposition is the same as the one for the binary case.

\begin{them}
    The type $\mathrm{Biproduct} (A_i)$ for a finite family $(A_i)$ of abelian groups is contractible.
\end{them}

\begin{proof}
    We want to construct an element of this type, but this type is a proposition, so as a finite set $J$ is a
    set $\langle J \rangle$ with a truncated equivalent $e : \|\langle J \rangle \simeq \mathrm{Fin}\;n\|_1$
    for some $n : \mathbb N$, we can proceed by induction on this truncated equivalence (i.e. equality by
    univalence) and construct an inhabitant only for the case of $J = 1,\ldots,n$. But in this case, we can
    take the following construction :
    \begin{itemize}
        \item The biproduct of an empty family is $0_\mathrm a$, the initial and final element of \textbf{Ab}.
        \item The biproduct $\displaystyle\bigoplus_{j=1}^{n+1} A_j$ is $\displaystyle
        \left(\bigoplus_{j=1}^n A_j\right)\oplus A_{n+1}$.
    \end{itemize}
\end{proof}

\subsection{The codiagonal}

Now we want to sum elements of a same group. For this, we use the structure of biproduct to take the pairing
on one side, to construct a function from the family $(x_i)$ to an element of $\bigoplus_{j\in J} X$ and the
codiagonal on the other side to map this element to an element of $X$.

\begin{defi}
    Let $J$ be a finite set and $(A_j)$ be a finite family of abelian groups over $J$. We define the space
    $\Pi A_j$ as the type $$\prod_{j : J} A_j$$ of families of elements of $A_j$ with the point $\lambda j.
    (a_j)_0$. This space is an abelian group and we have a natural family of functions 
    $$\prod_{j : J}\Pi A_j \toa A_j$$ which gives rise to the function $$\langle\_\rangle :\Pi A_j \toa 
    \bigoplus_{j\in J} A_j$$
\end{defi}

\begin{defi}
    Let $J$ be a finite set and $X : \textbf{Ab}$. We define $$\nabla^J_X : \bigoplus_{j\in J} X \toa X$$ by
    $$\nabla^J_X :\equiv [\idd_\mathrm a X,\ldots,\idd_\mathrm a X]$$
\end{defi}

Combining the two functions, we have the expected sum:

\begin{defi}
    Let $X : \textbf{Ab}$ and $J$ a finite set. We define the function $\Sigma^J_X$ as follows:
    \begin{align*}
        &\Sigma_J^X : \Pi X \toa X\\
        &\Sigma_J^X\;(x_j)_{j\in J} :\equiv \nabla^J_X (\langle (x_j)_{j\in J}\rangle)
    \end{align*}
\end{defi}

\section{Definition of the orientation}

We now combine the results of the previous two sections.

\begin{defi}
    Let $F$ be a finite set. We define the decidable powerset of $F$ and its subset of subsets of size $2$:
    \begin{align*}
        &\mathbb P^d : \mathrm{FinSet} \to \mathcal U\\
        &\mathbb P^d\;F :\equiv F \to \btwo\\
        &\mathbb P^d_2 : \mathrm FinSet \to \mathcal U\\
        &\mathbb P^d_2\;F :\equiv \sum_{P : \mathbb P^d\;F} \| \mathrm{fiber}\;P\;1 \simeq \btwo\|_1
    \end{align*}

    Both $\mathbb P^d$ and $\mathbb P^d_2$ are finite sets for $F$ finite.
\end{defi}

\begin{defi}
    The orientation of a finite set $F$ is the following torsor :
    \begin{align*}
        &\mathrm{or} : \mathrm{FinSet} \to \twotors\\
        &\mathrm{or}\;F :\equiv \Sigma_{\twotors}\;(\mathbb P^d_2\;F)
    \end{align*}
\end{defi}

This leads to the following definition of the sign of a permutation.

\begin{defi}
    We define $\mathfrak S_n :\equiv \mathrm{Fin}\;n = \mathrm{Fin}\;n$ and the following function :
    \begin{align*}
        &\varepsilon : \mathfrak S_n \to \btwo\\
        &\varepsilon\;\sigma = \mathrm{actToBool}(\mathrm{ap}_{\mathrm{or}}\;\sigma)
    \end{align*}
    Where $\mathrm{actToBool}$ is the function which, given an action on a torsor, returns its corresponding
    boolean.
\end{defi}

\end{document}